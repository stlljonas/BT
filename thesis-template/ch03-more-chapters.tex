\chapter{Your Central Work}

\section{Definitions and Problem Formulation}

    In this section describes the relvant physical definitions from which the robustness measure is derived and which will be referred to in the later code implementation.

    Examples will be given n terms of laikag quadruped robot as most of the testing of the framework was done with that. 

\subsection{Dynamical Systems}
    
    Dynamical systems are distinguished by an evolution of their state $\mathbf{x}(t)$ through time. This evolution can be fully described by a set of ordinary differential equations of the form $\dot{\mathbf{x}}(t) = \mathbf{F}(\mathbf{x}(t),t})$ which simplifies to $\dot{\mathbf{x}}(t) = \mathbf{F}(\mathbf{x}(t))$ under the assumption that the dynamical system is autonomous, i.e. is not explicitly dependent on time. 

    When solving for the explicit solution $\mathbf{x}(t)$, an initial condition $\mathbf{x_0}=\mathbf{x}(t_0)$ is required, which is a system state at an initial time. For simplicity and without loss of generality for autonomous systems, we set $t_0 = 0$. With this the Initial Value Problem (IVP) can be formulated: \begin{gather} \label{eq:1} find \ \ \mathbf{x}(t) \\ s.t. \ \dot{\mathbf{x}}(t) = \mathbf{F}(\mathbf{x}(t)) \\ and \ \ \mathbf{x}(0) = \mathbf{x}_0 \end{gather}

    We denote the solution to the IVP as $\mathbf{x}(t,\mathbf{x}_0)$. It represents trajectory of the system state through time given an initial condition. The space in which this trajectory lies is spanned by all possible states \mathbf{x} and termed the state space. Note that any future state of the trajectoy $\mathbf{x}(\tau,\mathbf{x}_0)$ at time $\tau$ can be interpreted as an initial condition of the IVP itself. It turns out that the new solution coincides with the initial one, i.e. $\mathbf{x}(t,\mathbf{x}_0) = \mathbf{x}(t,\mathbf{x}(\tau,\mathbf{x}_0))$, which illustrates how any state of a trajectory $\mathbf{x}(t) \ \in \ \mathbf{x}(t,\mathbf{x}_0)$ is suffitient to describe trajectory itself. 
    
    Mechanical systems (which are the focus of this project) tend to be described in terms of so called generalized coordinates $\mathbf{q}(t)=\begin{pmatrix}q_1(t)&q_2(t)&\ldots&q_n(t) \end{pmatrix}^T$. They are the minimal set of coordinates needed to fully describe the position and orientation of all of the systems elements. Their dimension n cooincides with the number of degrees of freedom of the system. The related differential equations are of second order, depending on $\ddot{\mathbf{q}}(t)$ in addition to $\dot{\mathbf{q}}(t)$ and $\mathbf{q}(t)$. Simply put, this is due to their derivation by Newton's second method, where forces acting on the system are related to the second time derivative via $F = ma$. 


    Through an order reduction (Appendix) these differential equations can be cast into the reviously mentioned general form \ref{eq:1}, where $\mathbf{x}(t) = \begin{pmatrix}q_1(t)&q_2(t)&\ldots&q_n(t)&\dot{q}_1(t)&\dot{q}_2&\ldots&\dot{q}_n \end{pmatrix}^T$ (maybe just use vectorial forms of q and qdot).
    When discussing high level concepts we will refer to the system state $\mathbf{x}(t)$ for simplicity, while in the code implementation the generalized coordinates $\mathbf{q}(t)$ and its derivatives will be more relevant. Keep in mind that both are equivalent.

    This implies that in order to solve the IVP, initial conditions $\mathbf{q}_0$ and $\dot{\mathbf{q}}_0$ are required. We can also state that for a system with n degrees of freedom, $\mathbf{x}(t) \in \mathbb{R}^{2n}$. The state space spanned by the generalized coordinates and velocities is termed the phase space $\mathbf{x}(t) = \begin{pmatrix}\mathbf{q}(t)\\\dot{\mathbf{q}}(t)\end{pmatrix}$. It can be used for a generalizable qualitative analysis of the behaviour of nonlinear dynamical systems. 




    %Go over to mechanical systems, explain generalized coordinates and show how this means that the state has q and qdot. Appendix to order reduction. Say how q_0 and qdot_0 are suffitient an necessary to fully characterize the system state. State that the state space for this constellation is called the phase space (used often). Is very interesting for analysis





    %It's elements are coordinates  in a chosen set of coordinates. For convenience we choose the so called generalized coordinates $\mathbf{q}(t)=\begin{pmatrix}q_1(t)&q_2(t)&\ldots&q_n(t) \end{pmatrix}^T$, which describe the system configuration with a minimal amount of coordinates, the number of which coincided with the degrees of freedom of the system. The individual q might represent angles, positions along a specific direction or even along a constrained curve. 

    %to acquire the equations of motion one needs to solve a 2n dimensional system of differential equations of the general form 
    %\begin{align*}
    %\dot{\mathbf{q}}(t) = \mathbf{f}(\mathbf{q}(t),\dot{\mathbf{q}}(t))\\
    %\ddot{\mathbf{q}}(t) = \mathbf{g}(\mathbf{q}(t),\dot{\mathbf{q}}(t))
    %\end{align*}
    %with $\mathbf{f}$ and $\mathbf{g}$ being some general nonlinear functions under the assumtion that the system is autonomous. 
    %q(t) describes the evolution through time.
    %q(t) can only be analytically found in a few cases and quickly breaks down when contact forces and other nonlinearities come into play. 


    %q and qdot are suffitient to describe the system as they are precisely the initial conditions for solving for the trajectory in future time. 


    %This is something for the appendix
    %It is often much simpler to formulate the dynamics in terms of differential equations, which in the case of mechanical systems result in ordinary differential equations with the first and second time derivative of q. 
    qdot = f(q,qdot)
    qddot = g(q,qdot)

    order reduction:

    x = [q,qdot] => xdot = [qdot, qddot] = F(x) = [f(x), g(x)]


    q,qdot,qddot 


    %$\math

    %The trajectory of a system is the evolution of the state x through time starting from an initial state at an initial time $\mathbf{x}_0 = \mathbf{x}(t_0)$. Explicit for trajectories $\mathbf{x}(t) = \mathbf{F}(t)$ are generally hard to find which is why system dynamics are usually described in the form of an initial value problem using systems of differential equations $\dot{\mathbf{x}}(t) = \mathbf{F}(\mathbf{x}(t),t)$. $\dot{\mathbf{q}}(t) = \begin{pmatrix}\dot{q_1}(t)&\dot{q_2}(t)&\ldots&\dot{q_n}\end{pmatrix}^T$ is denoted as the vector of generalized velocities. 

    %For mechanical systems the 


    %For ease of notation we define the state $\mathbf{x}(t)$ of a system by the concatenation of the generalized coordinates and velocities.

    %$\mathbf{x}(t) = \begin{pmatrix}\mathbf{q}(t)&\dot{\mathbf{q}}(t)\end{pmatrix}^T$



    %the solution of x may depend on anarbitrary order of time derivatives, however it can always be reduced to a first order system of ode's. 

    %we look at dynamical systems of the form of ivps  


    %Here we assume that the differential equations do not explicitly depend on time and are therefore autonomous $\%mathbf{F}(\mathbf{x}(t),t) = \mathbf{F}(\mathbf{x}(t))$.   

    %define state and trajectory first.


    Explic 
    The dynamic of a system can often be described by a system on differential equations, 
    The dynamical systems we dealt were autonomous and nonlinear in nature. Autonomous as we did not 
    state, derivatives, trajectory

    state
    ivp
    generalized coordinates and their derivatives
    trajectories 




\subsection{Attractors and Convergence}
    

    From analysis of nonliner dynamics we know that in the phase space of a system, there usually exist sets of states which show an attracting behaviour. Once a trajectory reaches an element of such a set, all future states will also be part of it. (insert math formula)

    Define an Attractor  $ A\ =\ \{\ \{\mathbf{x}\ \in \mathbb{R}^{2n}}\ \mid\ }$

    These so called attractors come in two variants.
    If the attractor consists of only one state, it is called a fixed point. Fixed points come with the condition of 
    $\dot{\mathbf{x}}(t) = \mathbf{0} \forall t \in \mathbb{R}$. This means the state of the fixed point is unchanging and a trajectory with the fixed point as an initial condition is reduced to a point in the phase space. A classical example of this is the stable bottom position of a pendulum, where if it starts with zero velocity, it won't leave the stable position. This is not the case for any other position as gravity will act on the mass (exept for the inverted position, however this is practically not realizable). 

    In the general case with A consisting of more than one state, (reference above eq) is not true. Rather the trajectory is moving through the sets of A, visiting every element at some point in time and returning to it repeatedly in the future in a periodic fashion. These sets are called limit cycles. This name is derived from what happens to states that are close to A, but not part of it. 

    define basin of attraction. 

    note that there may be an arbitrary number of attractors. 

    Within bounded time intervals, the evolution of a state trajectory can be described by two main behaviours. Either it stays confined to a set of states   (also reference cell mapping algorithms which do exactly that).

    examples for attractors (strandbeest for periodic orbit) 


    The phase space is a particular instance of state spaces, restricted to the set of generalized coordinates $q_i$ and their corresponding generalized velocities $\dot{q_i}$, sometimes in form of generalized momenta (via scaling by mass or inertia). It encompasses all possible initial conditions of a mechanical system and solutions of of the ivp result in trajectories through the phase space, starting at 


    Starting with a system and an initial condition that is part of an attractor 

    Attractors are sets of states in the phase space
    In a general nonlinear dynamical system, the evolution of the state can be divided into two main events using the concept of attractors. Attractors are sets of states toward which the system trajectories will converge or diverge from. If the set only consists of one state it is called a fixed point, else a periodic orbit where the trajectory will visit all states of the set periodically if given enough time. 
    We propose methods for dealing with both cases. 
    The set of initial conditions in the phase space for which the resulting trajectories converge towards the attractor is called the basin of attraction. If an initial condition does not lie in this set, it's trajectory diverges from the attractor. 
    Formally, a trajectory converges to an attractor A if $\underset{t \rightarrow \infty}{lim}\mathbf{x}(t) \in A$ 

    When referring to a system being robust to a disturbance, it means that it's trajectory in the phase space will return to the specified attractor under the influence of that particular disturbance. 

    Underline that this concept of convergene or divergence is a very general way of saying wheather a system works as intended or doesn't. 

\subsection{ Parameter space and Disturbance space}
    
    
    I feel like this should be part of the robustness measure

    separate from "dynamics", focus on 

    dimension disturbance space equals d

    In reference they did it with the phase space  

    We actively separate phase space and disturbance space for the concept to be more versatile

    The disturbance space is defined as the set containing all combinations of disturbances, against which the robustness of the system is evaluated. The choice of disturbances in "REFERENCE" are the initial conditions in the phase space (reference to the section in related work or copy that here) as they can be interpreted as the direct result of external forces. In general however, the disturbances may be chosen arbitrarily and the resulting robustness measure will quantify robustness against these. Examples of this were initial tilting of a quadruped over pitch and roll axes and oscillations with the disturbance space spanning the amplitudes and frequencies. Tuple combinations are favoured because of their ease of visualization and mild computational power requirements.  

    "REFERENCE" chose the DS to coincide with the phase space, as elements of it can be directly interpreted as the result of external forces on the system. This choice is not always feasible nor necessary, as will be shown in the following sections. 

    It may coincide with the phase space as implemented in "REFERENCE", in which case disturbances can be interpreted as the effects of external forces acting on the system. This moves the system onto a different trajectory 
    , however it will be shown in the following sections, that this is not always feasible nor necesseary. Especially in systems with a large number of degrees of freedom n, the resulting disturbance space is 2n dimensional. 

    case DS = phase space: d

    dimension parameter space = p
    The two main variable spaces of relevance move are the parameter space (PS), where each dimension represents the value of a chosen system parameter. The goal of finding a robustness measure can be though of as $RM:\mathbb{R}^p \rightarrow \mathbb{R}$, i.e. a mapping from the p-dimensional parameter space to a scalar value. (really it also depends on the disturbance space) Finding the best system parameters with respect to the resulting robustness can be formulated as an optimization problem ...


    
    state space v phase space

\subsection{robustness Measure}

    The basic idea of robustness 

    First explain the basics of the REF implementation
    Then expand to general disturbance spaces with a pointer to the meaning of DS = phase space

    
    The goal aim of the robustness measure is to quantify the robustness of a system with a particular parameter constellation with respect to a set of distrubances. 
    In REF, these disturbances were chosen to be initial conditions sampled from the phase space. They have a nice physical interpretation. However instantaneous forces are not the only kind of disturbance. Here we will apply the concept of the minimal Radius in REF described to some general disturbance space. This may concide with or be a subspace of the phase space, but it must not necessarily be the case. At the same time, evolution of the state trajectory and convergence will still be evaluated in the phase space. 


    As in "REFERENCE", the basic idea is that the more disturbances the system can endure, i.e. it converges to the attractor, the more robust it is. Discretizing the DS and simply counting the total number of disturbances the system is robust to is impractiacal because of the possibly fractal nature of the boundary of the basin of attraction (ref) and the fact that the number of evaluations is $O((1/h)^d)$, meaning that higher resolution and a larger DS will drastically increase computational time. To remedy this issue, the in "REF" proposed minimal Radius is found which denotes the radius of the largest hypersphere that sill fits completely within the basin of attraction.

    In other words, any combination of disturbances which lies on the surface of the hypersphere with the minimal radius in the DS will result in a trajectory that converges to the attractor in the phase space. 

    Taking sampling initial conditions form the phase space as disturbances gives a nice physical interpretation and in "REF" this was denoted as general robustness (or was it?), however as a counterexample, periodic or constant disturbances are cleary not covered by this. 
    This is why we worked with general disturbance spaces that can be chosen to work for the particular application at hand. 
    Note that the trajectory will still lie in the phase space and convergence evaluation 



\section{Code Implementation}

\subsection{Framework Overview}


differential equations are implicitly represented as simulation objects in dde. Their parameters can be set. The state x as well. Use the solver to compute the trajectory under a disturbance for a given amount of timesteps. check if trajectory converges or diverges. Do this repeatedly with initial conditions sample
talk about how knowing the exact shape of the basin of attraction is not necessary for optimization of the robustness, with this and because of the additional work needed to implement the cell mapping algorithms, the method with full trajectories was chosen. For this we need a solver. boom. Great segue!

\subsection{Solver}

Finding an explicit analytical solution to the IVP is possible for simple cases, but very hard if not impossible for more complex systems. 
The solution to the IVP can be found using numerical solvers. This is not an explicit solution but the trajectory must be iteratively computed by integrating the differential equations over small time steps. 

Here we used dde ... write some stuff about dde.

Describe what dde puts out
    DDE provides q, qdot and qddot at every timestep when computing the trajectories. 


Starting with a sytem of ODE's and as corresponding set of initial states, the first step towards computing robustness measure is finding the resulting trajectories. This is achieved by numerically integrating the differential equations over small timesteps until either convergence is detected or the $t_max$ is reached. With the chosen solver it is imperative to keep this timestep constant when comparing different robustness measures. The time step has a direct influence on the solvers accuracy and by that onto the system dynamics. It is advised to find a value that is a sufficient compromise between accuracy and cumputational time and keeping this fixed from there on.

trajectories here are not continuous but sets of states for discrete timepoints $[t_1,t_2,\ldots, t_n]$. 


\subsection{Detecting Attractors}

In order to evaluate wheather a trajectory converges 
    
    General idea with similarity of states to attractorstate

    Fixed points

    fixed points are defined states where $\dot{\mathbf{x}}(t) = \mathbf{0} \quad \forall \ t\  >\  t_0 $, i.e. $\mathbf{q}(t)=\mathbf{q}_{fp}, \dot{\mathbf{q}}(t) = \ddot{\mathbf{q}}(t)  \forall t > t_0$

    Periodic Orbits
        assume that period of attractor is the same as forcing frequency. Think controller forcing a gait given period, then the leg will also move at that period. This is not generally true and needs to be verified. 
        Poincare => issues with high computational effort and issues with systems that exhibit oscillations with a multitude of periods. 

    Issue with wrong attractor

    Simplification


\subsection{"Evaluating" Convergence}
    Cleary it is impractical to let t go to infitiy, which is why some $t_n$, max needs to be defined at which the simulation stops. 
    quite similar to 

\subsection{minRad algorithm}

    maybe not go into too much detail on how it works (look at ref)

    tuning

    visualizations

\subsection{boundarys PS DS}

    

    elephant in the room 
    changing even the unit changes the scaling and therefore the robustness value.

    explain and give examples on how boundaries were chosen. 

    was noted in REF that it can be an ellipse as well, but they didn't know how to choose it's parameters. We do this by analysis and finding reasonable boundaries in the DS.

    order of complexity depending on PS and DS

\subsection{Multithreading}

    computational time can be reduced by 
    pseudocode

\subsection{Application to specific systems}
    
    analysis needed to find parameter space and disturbance space boundaries and good initial guesses.
    analysis of high dof motion tricky (bad 3d image), rather plot every coordinate over time (image). 

    applyParameters

    simAndEval

\subsection{Optimization and Complexity reduction}
    

    find out how the order of computational effort when making discretization smaller vs how much one bisection algorithm iteration can do. 
    cmaes
    given robustness value, any optimizer that does not depend on derivatives can be used. 
    Of course one can also just explore the entire parameter space, howeve that is quite costly. Useful for debugging though (rough estimate if optimizer converges).

    complexity reductions should probably be noted where they are implemented (i.e. not in this section)

\section{Tests}


    Introduction to laikago
        high dof
        rather long computational time. 
    optimization of robustness examples

    Laikago Droptest
    Laikago Swingtest
