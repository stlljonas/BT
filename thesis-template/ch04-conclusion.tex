\chapter{Conclusion and Outlook}

%(recap quickly)

In this project a measure of robustness with respect to specific sets of disturbances was implemented and tested. It was adapted from an approach solving the initial value problem associated with th isolated system to one taking into account disturbances and the environment as well. Simulations of the disturbed system were perfomed with a provided dynamics engine. Optimization of the parameters of a quadruped robot model were performed in two tests, showing clear improvements of robustness with respect to the chosen sets of disturbances. 
For every new test, the number of required modificaions to the framework was kept small, however considerations about the application of parameters and disturbances still need to be undertaken on a case by case basis. 

An inherent dependency of the robustness measure on the choice of disturbances remains, strongly restricting the generality of the optimized parameters. Of the set of all imaginable disturbances however, only few will realistically occur in the real world, making the strive for general robustness measure questionable in the context of complex systems. Future work may investigate the results achieved by optimizing for larges sets of different types of disturbances simultaneously. Further tests with attractors as limit cycles may also be performed as the tests were limited to fixed points. 

The choice of parameters opens up additional possibilities of improvement. While in this work the motor parameters were chosen because of the ease of implementation, changing physical dimensions could yield larger increases in robustness. Intiuitively, a quadruped with a wider core and shorter legs should be more robust in both the drop and swing tests, which could be verified with this method. The maximal theoretical robustness in the performed test was inherently limited, as there exist hard limits for the maximal disturbances from which the system could theroetically recover. A rotation of either the robot or the floor by 90 degrees could not possibly be compensated, even if active control of the model was utilized. 

The question of the scaling of the disturbance space axes is left open. Here it was coped with by finding balanced choices, but it may also present the possibility of active control over the optimization process.  

Ultimately, the available computational power is still the main limiting factor for large scale applications. Optimization of the framework or the solver w.r.t. computational time is a promising starting point for expanding on the number of possible applications of the method. 

%No tests were performed on limit cycles. Optimizing parameters for a robot with a periodic walking gait would be very interesting (rephrase). 


%Additional optimization of the framework itself could also enable the anlysis and optimization of 


%Regarding optimization, with more sophisticated optimizers or just better tuning of the used algorithm, further improvements in the results and computational speed could be achieved.   





% explanation why there can't be that much optimization for swing and droptest

%Implication that the system must be dynamical in nature (i.e. it must evolve). So a rudimentary control strategy must be implemented or outside forced must be applied. Don't quite know where to put this. 


%further explore effects of combining disturbences of differnent sorts and the effects of the choice of bounds.

% to systems of high dof with full phase space. Is it even possible? How much can the code be optimized?

%apply to physical dimensions of systems (intuitively more drastic effects)

%using the phase space as the disturbance space but heavily restricting it (actuator limits, improbable constellations etc)

%Optimizing with limit cycles as attractors. 

%solver (or rather finding the trajectories in general) will always be the largest bottleneck and finding methods to reduce number of trajectories to be evaluated or the computational time per trajectory would be very benefitial


%Seems unlikely that this will ever be plug and play. Lots of tuning, lots of fiddling around. But there could definitely be applications, especially in novel systems where rigorous analysis is lacking. 

%High complexity is still an issue. 

%also parallelize the robustness measure computation, as cmaes alwayse nneds to evaluate a batch of robustness measures at once. 